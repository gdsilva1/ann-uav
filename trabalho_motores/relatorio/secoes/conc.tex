\chapter{Conclusão}

A realização dos ensaios de fluxo no cabeçote do motor EA211 TSI e do ensaio dinamométrico no motor John Deere 6068 Tier 3 proporcionou uma análise abrangente e detalhada do desempenho e das características operacionais dos motores de combustão interna. 
No ensaio de fluxo no cabeçote, os cálculos e gráficos proporcionaram a compreensão sobre o comportamento do fluxo em relação ao levante da válvula, tanto para a admissão na válvula de admissão, escape na válvula de admissão e escape na válvula de escape. 
A análise da área de cortina em função do levante da válvula revelou um aumento linear, enquanto os coeficientes de descarga, considerando a área de garganta e área mínima, ofereceram informações importantes sobre a eficiência do escoamento e as limitações relativas à geometria das válvulas.
No ensaio dinamométrico, a estabilidade do torque foi em torno de \SI{670}{N.m}, a constância do BMEP a partir de 1600 RPM e a estabilização do boost em 180 kPa. A análise da velocidade média e máxima, com a carga lateral do pistão, proporcionou uma compreensão aprofundada do comportamento dinâmico do motor em diversas condições de operação. A interação entre os parâmetros, como torque, BMEP, pressão de boost e força lateral do pistão, evidenciou a importância da sintonia fina no projeto do motor para garantir desempenho uniforme e eficiência térmica. A estabilidade da força lateral do pistão contribuiu para uma operação suave e eficiente, minimizando vibrações e desgaste mecânico. O índice de Mach indica torque máximo em 0.59 e potência máxima em 2.7. Assim, a abordagem integrada desses elementos essenciais, aliada à precisão nos controles, confirma a capacidade do motor em proporcionar uma operação eficiente e confiável em diversas condições.
Portanto, os resultados e discussões apresentados neste trabalho demonstram a habilidade de conduzir uma análise detalhada de um motor de combustão interna, semelhante à abordagem realizada por uma montadora. 