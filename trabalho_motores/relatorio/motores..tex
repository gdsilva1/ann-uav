\documentclass[
    12pt,                % tamanho da fonte
    brazil               % idioma
    openany,             % capitulos podem comecar em qualquer pagina
    oneside,             % para impressao de um unico lado da folha
    chapter=TITLE,       % deixa a letra dos capitulos em maiuscula
    % sumario=tradicional, % estilo do sumario
    % fleqn
]{abntex2}
%
% ===============
%    PACKAGES    
% ===============
%
\usepackage[version=4]{mhchem}
\usepackage[scaled]{helvet}  % fonte nao serifada 
\usepackage{stix2}           % fonte serifada
\usepackage{graphicx}        % permite inserir imagens 
% \usepackage{microtype}      % melhora da tipografia (quebra o sumario)
\usepackage[T1]{fontenc}     % define melhorias para o tipo da fonte
\usepackage[utf8]{inputenc}  % codificacao do texto
\usepackage{indentfirst}     % identa o primeiro paragrafo
\usepackage{tabularray}      % permite criacao de tabelas personalizadas
\usepackage{xcolor}
\usepackage[locale=DE]{siunitx}
\usepackage[
    capitalise,              % nome das fig/tab/ em letras maiusculas
    nameinlink               % hyperref no nome das fig/tab
]{cleveref}
\usepackage{csquotes}
% nas versoes atuais do texlive, o sumario nao compila como deveria,
% entao a formatacao deve ser feita manualmente
% \usepackage{tocloft}
% \cftsetindents{part}{0em}{\cftlastnumwidth}
% \cftsetindents{chapter}{0em}{\cftlastnumwidth}
% \cftsetindents{section}{0em}{\cftlastnumwidth}
% \cftsetindents{subsection}{0em}{\cftlastnumwidth}
% \cftsetindents{subsubsection}{0em}{\cftlastnumwidth}
% \cftsetindents{paragraph}{0em}{\cftlastnumwidth}
% \cftsetindents{subparagraph}{0em}{\cftlastnumwidth}
% \renewcommand{\cftchapterfont}{\bfseries\MakeUppercase}
% \renewcommand{\cftchapterpagefont}{\normalsize\cftchapterfont}
% \renewcommand{\cftsectionfont}{\bfseries}
% \renewcommand{\cftsectionpagefont}{\cftsectionfont}
\renewcommand{\cftsubsectionfont}{\normalfont\sffamily}
\renewcommand{\cftsubsectionpagefont}{\cftsubsectionfont}
%
% ==================
%    REFERENCIAS    
% ==================
%
\usepackage[
    style=abnt,
    ittitles
]{biblatex}
\addbibresource{ref.bib}
%
% ==================
%    DOC OPTIONS    
% ==================
%
% as configuracoes a seguir formatam o estilo dos headings
\renewcommand{\ABNTEXchapterfont}{\sffamily\bfseries} % estilo do capitulo
\renewcommand{\ABNTEXchapterfontsize}{\Large}         % tamanho fonte capitulo
\setsecheadstyle{\sffamily\bfseries\large}            % estilo da secao
\setsubsecheadstyle{\sffamily\large}                  % estilo da subsecao
\setsubsubsecheadstyle{\sffamily\large}               % estilo da subsubsecao
\setlength{\afterchapskip}{\baselineskip}             % espacamento abaixo cap.
\UseTblrLibrary{booktabs}                             % para tabularray
\hypersetup{
    colorlinks=true,
    allcolors=blue!50!black,
    pdftitle={\@title},
    pdfauthor={\@author},
    pdfsubject={\imprimirpreambulo}
}
\captionnamefont{\footnotesize\sffamily\bfseries} % formatacao do titulo das figuras
\captiontitlefont{\footnotesize} % formatacao do texto do titulo das figuras
\newcommand{\ma}{\text{Ma}}
%
% ===========
%    CAPA    
% ===========
%
\renewcommand{\imprimircapa}{%
    \begin{center}\sffamily
        \includegraphics{./figuras/feis_unesp.pdf}
        \vfill
        {\bfseries\large AVALIAÇÃO DO MOTOR JOHN DEERE 6068 TIER 3}
        \vfill
        \begin{tabular}{ll}
            Nome & RA \\ \hline
            Alícia Ramos Modesto                  & 181052725 \\
            Gabriel Duarte da Silva               & 182054047 \\
            Higor Balsarini                       & 182055302 \\
            Matheus Henrique Panini               & 182053857 \\
            Yuri Fernando Oliveira Kazitani Cunha & 231052669
        \end{tabular}
        \vfill
        Docente: Gabriel Coelho Rodrigues Alvares
        \vfill
        Ilha Solteira -- SP \\
        Dezembro de 2023
    \end{center}
    \clearpage
}
%
\begin{document}
%
\imprimircapa\clearpage    % adicao da capa
\chapter*{}
\listoffigures\clearpage   % adicao da lista de figuras
\listoftables\clearpage    % adicao da lista de tabelas
\tableofcontents\clearpage % adicao do sumario

\textual

\chapter{Results and Discussion}\label{sec:results}

% \begin{figure}
%     \centering
%     \import{/home/gabriel/Documentos/deep-learning/report/figures/}{plot_style.pgf}
% \end{figure}
\chapter{Metodologia}

\section{Revisão da Literatura}

Os conceitos teóricos explicados a seguir são baseados nas obras de \textcite{heywood2018internal, ferguson2015internal} e nas notas de aula de

\subsection{Áreas}

\subsection{Coeficiente de Descarga}

O coeficiente de descarga é a razão entre o fluxo de ar que está passando através do componente durante o ensaio, pelo fluxo de ar que deveriapassar pelo componente durante o ensaio caso o escoamento isentrópico.
%
\begin{equation}
    C_d = \frac{V_r}{V_t}
    \label{eq:coef_descarga}
\end{equation}
%
onde $V_r$ o fluxo de massa real e $V_t$ é o fluxo de massa caso o escoamento fosse isentrópico.

\subsection{Pressão Média}

A pressão média indicada (IMEP) é aquela determinada baseada na geometria do motor e considerando as transformações do ciclo como irreversiveis. 
A pressão média de atrito (FMEP) indica a pressão perdida devido ao atrito e o bombeamento.
A pressão média efetiva (BMEP) é a pressão que o motor de fato tem trabalho líquido.
A BMEP pode ser calculada como mostra a seguir:
%
\begin{equation}
    \text{BMEP} = \frac{120 \dot{W}_b}{n V_d N}
    \label{eq:bmep}
\end{equation}
%
onde $\dot{W}_b$ e a potência, $n$ é quantidade de cilindros do motor, $V_d$ e o volume deslocado e $N$ é a rotação do motor em RPM.

\subsection{Carga Lateral}

A carga lateral do pistão é obtida através da análise de forças no pistão a partir das leis da mecânica dos sólidos e é dada por:
%
\begin{equation}
    F_L = \frac{\pi d P (R/L) \sin\alpha}{4\sqrt{1-(R/L)^2\sin^2\alpha}}
    \label{eq:carga_lateral}
\end{equation}
%
onde $d$ é o diâmetro do pistão, $P$ é a pressão exercida, $R$ é o comprimento da biela, $L$ é o comprimento da manivela e $\alpha$ é o ângulo percorrido pela manivela a partir do ponto morto superior (PMS).

\subsection{Velocidade do pistão}

A partir de geometria simples, é possível determinar a posição do cilindro em relação ao ângulo de rotação da manivela. Sabendo que $\alpha=\omega t$, determina-se a velocidade através da derivada da posicão do cilindro:
%
\begin{equation}
    v(t) = R \omega \sin{\left(\omega t \right)} + \frac{R^{2} \omega \sin{\left(\omega t \right)} \cos{\left(\omega t \right)}}{L \sqrt{1 - \dfrac{R^{2} \sin^{2}{\left(\omega t \right)}}{L^{2}}}}
    \label{eq:velocidade}
\end{equation}

\subsection{Número de Mach}

O número de Mach é um termo adimensional definido pela seguinte expressão:
%
\begin{equation}
    \ma = \frac{V}{c} = \frac{\text{velocidade do escoamento}}{\text{velocidade do som}}
    \label{eq:mach}
\end{equation}
%
que descreve a velocidade do escoamento. Quando $\ma = 1$ o escoamento é considerado sônico; quando $\ma < 1$, subsônico; quando $\ma > 1$, supersônico e quando $\ma \gg 1$, hipersônico \cite{cengel2015mecanica}.
\section{Montagem Experimental}

\section{Procedimento Experimental}
\chapter{Resultados}

\section{Resultados medidos}

As condições iniciais proposta foram: (i) para o ensaio de fluxo no cabeçote do motor EA211 TSI, a pressão é de $25 \text{ pol \ce{H2O}}  \approx  \SI{6,2271}{kPa}$, variando a abertura das válvulas entre 0 e 10 mm, com passo de 1 mm; (ii) para o ensaio dinamométrico, o motor avaliado é um John Deere 6068 Tier 3, com condição de carga de 75\%.

\section{Resultados calculados}

\subsection{Ensaio de Fluxo no Cabeçote}

\subsubsection*{Áreas}

\subsubsection*{Coeficiente de descarga}


\subsection{Ensaio Dinamométrico}

\subsubsection*{Pressão Média}

\subsubsection*{Torque}

\subsubsection*{Potência}

\subsubsection*{Carga Lateral}

\subsubsection*{Velocidade}

\subsubsection*{Pressão de Boost}

\subsubsection*{Índice de Mach}
\chapter{Conclusão}

A realização dos ensaios de fluxo no cabeçote do motor EA211 TSI e do ensaio dinamométrico no motor John Deere 6068 Tier 3 proporcionou uma análise abrangente e detalhada do desempenho e das características operacionais dos motores de combustão interna. 
No ensaio de fluxo no cabeçote, os cálculos e gráficos proporcionaram a compreensão sobre o comportamento do fluxo em relação ao levante da válvula, tanto para a admissão na válvula de admissão, escape na válvula de admissão e escape na válvula de escape. 
A análise da área de cortina em função do levante da válvula revelou um aumento linear, enquanto os coeficientes de descarga, considerando a área de garganta e área mínima, ofereceram informações importantes sobre a eficiência do escoamento e as limitações relativas à geometria das válvulas.
No ensaio dinamométrico, a estabilidade do torque foi em torno de \SI{670}{N.m}, a constância do BMEP a partir de 1600 RPM e a estabilização do boost em 180 kPa. A análise da velocidade média e máxima, com a carga lateral do pistão, proporcionou uma compreensão aprofundada do comportamento dinâmico do motor em diversas condições de operação. A interação entre os parâmetros, como torque, BMEP, pressão de boost e força lateral do pistão, evidenciou a importância da sintonia fina no projeto do motor para garantir desempenho uniforme e eficiência térmica. A estabilidade da força lateral do pistão contribuiu para uma operação suave e eficiente, minimizando vibrações e desgaste mecânico. O índice de Mach indica torque máximo em 0.59 e potência máxima em 2.7. Assim, a abordagem integrada desses elementos essenciais, aliada à precisão nos controles, confirma a capacidade do motor em proporcionar uma operação eficiente e confiável em diversas condições.
Portanto, os resultados e discussões apresentados neste trabalho demonstram a habilidade de conduzir uma análise detalhada de um motor de combustão interna, semelhante à abordagem realizada por uma montadora. 

\printbibliography
	
\end{document}