\chapter[Introduction]{Introduction}\label{cha:intro}

\gls*{uav} has been used for several applications, such as entertainment, sports transmissions and commercial applications.
However, historically, it was primarily designed to achieve military goals, including unmanned inspection, surveillance, reconnaissance, and mapping of inimical areas.
Over time, its applications extended to other areas, like geomatics, for data collecting through photogrammetry. 
This way, collecting images using \gls*{uav} provides a bunch of applications in the aerial close-range domain, making it a low-cost alternative to the traditional manned aerial photogrammetry for mapping or detailed 3D recording information and being a valid complementary solution to terrestrial acquisitions \citep{nex_uav_2014}.

In Brazil, \gls*{uav} is widely used in agricultural situation, therefore, tracking, monitoring and collecting information in real time from remote areas for agriculture and livestock are quite relevant. 
\citet{abade_construcao_nodate} showed the development and construction of an \gls*{uav} able to board remote sensing applications with images and radio frequency for this purpose. 
Still with the same bias, but targeting another aspect, \citet{otake_produtos_nodate} used \gls*{uav} to generate cartographic products for agriculture purpose. 
The main goal was to detect failure of planting, the projection of contour lines and the elaboration of use of soil map.

In such manner, the use of \gls*{uav} equipped with cameras to access places where human access might be difficult is a way to spend less effort in many contexts, decreasing the chances of accident and spare financial recourses.
\citet{dadrasjavan_automatic_2019} considered the use of \gls*{uav} useful for acquiring reliable information about the pavement of the road and monitoring any kind of crack on it by selecting key frames and generating ortho-image.
Non road regions in the scene are discarded and crack elements are extracted.
Subsequently, through \gls*{svm} classification true cracks are detected. 
On the other hand, \citet{sushant_localization_2017} used a MATLAB\textsuperscript{\textregistered} implementation to both localizing the position of the \gls*{uav} and detect cracks in railway tracks. For the first goal, a \gls*{mcl} method was carried out for the software senses its position and for the second goal, and software method was able to detect the railway cracks comparing the intensity of the color of the image. \citet{lesiak_inspection_2020} also bring the attention to the inspection and
maintenance of railway infrastructure with the use of \gls*{uav} and its implementation.

Hence, it's clear the demand of \glspl*{uav} in various sceneries, in special to assist detecting damages in specific types of structures. Not only understand and determine the method is crucial, but also the way that it's implemented and its viability.

\section{Objective}\label{sec:objective}

To develop a process to obtain and analyze aerial images to detect railway cracks.





