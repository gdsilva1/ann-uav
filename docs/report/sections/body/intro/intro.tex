\chapter[Introduction]{Introduction}\label{cha:intro}

\gls*{uav} has been used for several applications, such as entertainment, sports transmissions and commercial applications.
However, historically, it was primarily designed to achieve military goals, such as unmanned inspection, surveillance, reconnaissance, and mapping of inimical areas.
Over time, its applications extended to other areas, like geomatics, for data collecting through photogrammetry. 
This way, collecting images using \gls*{uav}, provides a bunch of applications in the aerial close-range domain, making it a low-cost alternative to the traditional manned aerial photogrammetry for mapping or detailed 3D recording information and being a valid complementary solution to terrestrial acquisitions \cite{nex2014uav}.

In Brazil, \gls*{uav} is widely used in agricultural situation, therefore, tracking, monitoring and collecting information in real time from remote areas for agriculture and livestock are quite relevant. 
\textcite{abade2016construccao} showed the development and construction of an \gls*{uav} able to board remote sensing applications with images and radio frequency for this purpose. 
Still with the same bias, but targeting another aspect, \textcite{otake2017produtos} used \gls*{uav} generating cartographic products for agriculture purpose. 
The main goal was to detect failure of planting, the projection of contour lines and the elaboration of use of soil map.

In such manner, the use of \gls*{uav} equipped with cameras to access places where human access might be difficult is a way to spend less effort in several applications, decreasing the chances of accident and spare financial recourses.
\textcite{dadrasjavan2019automatic} considered the use of \gls*{uav} as useful for acquiring reliable information about the pavement of the road and monitoring any kind of crack on it. On the other hand, \textcite{sushant2017localization} used a MATLAB\textsuperscript{\textregistered}
implementation to both localizing the position of the \gls*{uav} and detect cracks in railway tracks.

