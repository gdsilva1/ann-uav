% ===============
%    PACKAGES    
% ===============
\usepackage{amsmath,amsfonts,amssymb,amsthm}
\usepackage{stix2}
\usepackage[semibold,proportional]{sourcesanspro}
\usepackage{graphicx}                               % permite inserir imagens 
\usepackage{svg}                                    % permite inserir svg
\usepackage{xcolor}                                 % habilita as cores
\usepackage{siunitx}                                % unidade de medidas de forma descomplicada
\usepackage{subcaption}                             % permite colocar subfiguras
% \usepackage{microtype}                            % melhora da tipografia
% \usepackage[T1]{fontenc}                            % define melhorias para o tipo da fonte
% \usepackage[utf8]{inputenc}                         % codificacao do texto
\usepackage{indentfirst}                            % identa o primeiro paragrafo
\usepackage{tabularray}                             % permite criacao de tabelas personalizadas
\usepackage{algorithm}
\usepackage{algpseudocode}
\usepackage[
    capitalise,                                     % nome das fig/tab/ em letras maiusculas
    nameinlink                                      % hyperref no nome das fig/tab
]{cleveref}
% ===============
%    DOC INFO    
% ===============
\titulo{Artificial Neural Networks applied to Unnamed Aerial Vehicle Control}
\autor{Gabriel Duarte da Silva}
\local{Ilha Solteira, SP}
\data{2023}
\instituicao{%
    São Paulo State University (UNESP)
    \par
    School of Engineering of Ilha Solteira
    \par
    Mechanical Engineering Department
}
\tipotrabalho{Iniciação Científica}
\preambulo{Undergraduate research report for Group of Intelligent Materials and Systems (GMSINT).}
\orientador{Douglas D. Bueno}
% ==================
%    DOC OPTIONS    
% ==================
% as configuracoes a seguir formatam o estilo dos headings
\renewcommand{\ABNTEXchapterfont}{\sffamily\bfseries} % estilo do capitulo
\renewcommand{\ABNTEXchapterfontsize}{\large}         % tamanho fonte capitulo
\setsecheadstyle{\sffamily\bfseries\large}            % estilo da secao
\setsubsecheadstyle{\sffamily\large}                  % estilo da subsecao
\setsubsubsecheadstyle{\sffamily\large}               % estilo da subsubsecao
\setlength{\afterchapskip}{\baselineskip}             % espacamento abaixo cap.
\UseTblrLibrary{booktabs}                             % para tabularray
\hypersetup{
    colorlinks=true,
    allcolors=blue!50!black,
    pdftitle={Artificial Neural Networks applied to Unnamed Aerial Vehicle Control},
    pdfauthor={Gabriel D. Silva},
    pdfsubject={\imprimirpreambulo},
    pdfcreator={Gabriel D. Silva},
    pdfkeywords={neural networks, control, unmanned aerial vehicle}
}
\captionnamefont{\footnotesize\sffamily\bfseries}
\captiontitlefont{\footnotesize}
\definecolor{magenta}{rgb}{1,0,1}
\colorlet{rosa}{magenta!10}
% configuracoes do pseudocodigo (Adam)
\algrenewcommand\algorithmicwhile{\textsf{\textbf{\textsf{while}}}}
\algrenewcommand\algorithmicrequire{\textsf{\textbf{\textsf{Require: }}}}
\algrenewcommand\algorithmicdo{\textsf{\textbf{\textsf{do}}}}
\algrenewcommand\algorithmicend{\textsf{\textbf{\textsf{end}}}}
\algrenewcommand\algorithmicreturn{\textsf{\textbf{\textsf{return}}}}
% ==================
%    REFERENCIAS    
% ==================
\usepackage{csquotes}
\usepackage[
    style=abnt,
    justify,
    indent,
    giveninits,
    extrayear,
    noslsn,
    ittitles
]{biblatex}
\addbibresource{ref.bib}