\chapter{Conclusion}\label{sec:conclusion}

The use of AI is widespread in various areas, reaching people at different levels. 
Its application in numerous fields makes it useful to solve problems with the vast amount of data collected nowadays.
Using AI to achieve goals is a good strategy. 
Controlling UAVs is vital because it can reach difficult areas with ease.
Keeping them on the desired trajectory is also crucial to guarantee flight stability and ensure that goals are not interrupted by external disturbances. 
A modern and effective strategy involves the use of ANNs to determine UAV forces and keep them on the right trajectory.
As observed, a simple ANN with small samples trained with limited resources was able to predict force patterns and satisfactory results.
Therefore, the use of AI with ANN techniques is a good strategy to ensure flight maintenance, and improving the ANN with more samples in data generation can enhance its capability to determine UAV control forces with greater precision.

\section{Next Steps}

Although the developed ANNs yielded satisfactory results, here are some suggestions for future work:

\begin{itemize}
    \item use more samples to train the ANN;
    \item try different architectures for each ANN;
    \item split different segments of the trajectory among different ANNs;
    \item utilize the forces predicted by the ANNs in the white box script and verify if the flight is executed appropriately;
    \item use different out-of-scope deep learning techniques and try traditional machine learning techniques for well-structured data.
\end{itemize}