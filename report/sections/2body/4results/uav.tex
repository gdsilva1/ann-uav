\section{Unnamed Aerial Vehicle}

\subsection{Comparison of the Model with the Script}

The~\cref{fig:uxt_comparison_model_script} shows the comparison of the control torque from the script with the \gls*{nn} created model.
%
\begin{figure}[!htb]
    \centering
    \caption[Comparing the model with the script]{Comparing the model with the script. The continuous curve is the control torque from the script, while dashed curve is the output from the \gls*{nn} model.}
    \import{/home/gabriel/Documentos/deep-learning/report/figures/4results/uav}{uxt_comparasion_model_script.pgf}

    {\footnotesize Source: prepared by the author.}
    \label{fig:uxt_comparison_model_script}
\end{figure}

From the model statistics, the loss function for the test sample gave the result of 0.00091, which is an acceptable value for its purpose.
Although the~\cref{fig:uxt_comparison_model_script} looks like to show some discrepancy for \(U_2(t)\) and \(U_3(t)\), they do not mean the model did not predict precisely the control torque. 
After the first 10 seconds, which is the time that the \gls*{uav} is leaving the ground, the model is able to describe the control torque very well.
The major error, i.e., in the beginning of the motion may be caused by the interference of the \(z\)-axis trajectory. The discrepancy, actually, represents very little in the major context, since the image scale may distort the real values.