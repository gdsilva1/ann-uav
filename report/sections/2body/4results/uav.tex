\section{Study Case 1: Unnamed Aerial Vehicle}

\subsection{Model Summary}

After the \gls*{nn} training, it was tested using the 200 samples, as the~\cref{tab:ann_parameters_uav}.
The metrics to evaluate the model are shown in the~\cref{tab:model_summary_uav}.
%
\begin{table}[!htb]
    \centering
    \caption{Model summary}
    \begin{tblr}{
        row{even}={light_color}
    }
    \toprule
    Parameter & Value \\
    \midrule
    MSE & 0.00085 \\
    MAE & 0.00746 \\
    \bottomrule
    \end{tblr}

    {\footnotesize Source: prepared by the author.}
    \label{tab:model_summary_uav}
\end{table}

\begin{figure}[!htb]
    \centering
    \caption{Some caption}
    \import{/home/gabriel/Documentos/deep-learning/report/figures/4results/uav/}{epochsxloss.pgf}
\end{figure}

\subsection{Comparison of the Model with the Script}

The~\cref{fig:uxt_comparison_model_script} shows the comparison of the control torque from the script with the \gls*{nn} created model for the same \(\symbf{x}_s\).
%
\begin{figure}[!htb]
    \centering
    \caption[Comparing the model with the script]{Comparing the model with the script. The continuous curve is the control torque from the script, while dashed curve is the output from the \gls*{nn} model.}
    \import{/home/gabriel/Documentos/deep-learning/report/figures/4results/uav}{uxt_comparasion_model_script.pgf}

    {\footnotesize Source: prepared by the author.}
    \label{fig:uxt_comparison_model_script}
\end{figure}

From the model statistics, the loss function for the test sample gave the result of 0.00085, which is an acceptable value for its purpose.
Although the~\cref{fig:uxt_comparison_model_script} looks like to show some discrepancy for \(U_2(t)\) and \(U_3(t)\), they do not mean the model did not predict precisely the control torque. 
After the first 10 seconds, which is the time that the \gls*{uav} is leaving the ground, the model is able to describe the control torque very well.

The major error, i.e., in the beginning of the motion may be caused by the interference of the \(z\)-axis trajectory. 
The discrepancy, actually, represents very little in the major context, since the image scale may distort the real values.
That said, other possible reason for the error in the model is the quantity of samples to make the training of the neural network. 
A thousand trajectories is not a good quantity for the model to make a good prediction.
To have an accurate model, it should have at least one hundred thousand trajectories, but due to the processing limitation, it was not possible.

Even though the model curve did not overlap the curve from the script, it gave the same pattern.