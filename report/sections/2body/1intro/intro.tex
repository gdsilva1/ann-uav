\chapter{Introduction}

The use of Artificial Intelligence (AI) is very prevalent nowadays \cite{lee2020,poola2017,rabunal2006}. 
This area of statistics neither is new nor started just now with the autonomous cars and voice assistants \cite{muthukrishnan2020}, but it is clear that in recent years, it has been increasingly gaining more popularity. 
This happens mainly because of the advances that the World Wide Web has had over the years \cite{leiner2009,cohen-almagor2011}, from dial-up internet connection back in the eighties to now, with broadband internet and smartphones equipped with 5G connection. 
Another factor is that in the past, the cost to obtain a large capacity of storage memory was significantly more expensive than it is now, making it cheaper and easier to acquire memory for storing information today \cite{goda2012}. 
With the amount of data available, the evolution of internet, and storage capacity, it is now not difficult to obtain, maintain and analyze databases to make decisions \cite{duan2019}.

AI application is everywhere and today, more than ever, it is easy to realize that. 
Whether it isto receive multimedia recommendations on streaming platforms, as is the case with Netflix, YouTube, Spotify, and many others \cite{chan-olmsted2019}, or to make predictions on the financial market and sports betting \cite{milana2021,kollar2021,hubacek2019}, AI is behind the scenes making all the magic happen. 
Evidently there is nothing truly magical about them; it is pure mathematics combined with a programming language that produces the algorithm capable of performing these tasks \cite{goodfellow2016,aurelien2022,raschka2015,raschka2022}.
The launch of ChatGPT--3, and shortly thereafter ChatGPT--4, has shown the power of those technologies and how they can change the way people do things \cite{biswas2023,biswas2023a, lund2023,baidoo-anu2023}.

In the application of smart systems, the use of AI is widespread in Structural Health Monitoring (SHM), which is heavily used in the aerospace and civil fields, \cite{azimi2020,ye2019}. 
The level and the complexity of the AI to be applied to monitor the structure, whether is going to use deep learning and artificial neural networks (ANNs) or simpler machine learning methods like regressions, are determined by the problem itself and the results desired \cite{farrar2012}. 
In some cases, standard methods using numerical techniques may not be feasible, especially when dealing with extensive data analysis. 
Thus, opting for the AI approach becomes an alternative to obtain the necessary monitoring results in a more practical manner \cite{smarsly2007,sun2020}. 

Still in this context, but in the field of Unmanned Aerial Vehicle (UAV), the use of AI can be combined to integrate UAV through wireless communication networks \cite{lahmeri2021}. This integration proves valuable in the agriculture sector \cite{ahirwar2019} incorporating technologies such as the Internet of Things \cite{verdouw2016,tzounis2017}.
Furthermore, AI applications can be subtle, such as utilizing a built-in MATLAB function to make a simple ANN to determine the final pose of a UAV based on the initial pose and the forces applied on it \cite{geronel2023}. It can be more sophisticated, involving the use of machine and deep learning algorithms to predict materials properties, design new materials, discover new mechanisms and control real dynamic systems \cite{guo2021,assilian1974}.

It is clear, therefore, that AI can transit into different fields, including entertainment, business, health care, marketing, financial, agriculture, engineering, among others \cite{ruiz-real2020,yu2018,davenport2019,verma2021,mhlanga2020,pannu2015,ghatrehsamani2023}. 
The use of Big Data can not only clarifies the scenario to be studied but also tupports the formulation of strategic decisions \cite{jeble2018,koscielniak2015}.
The improvement in internet and hardware capabilities \cite{baji2018}, alongside the ease of storing data at accessible costs, encourages the AI use due to the benefits it can provide.

\section{Motivation}

With the 4.0 industry, the engineering evolution is growing bigger every year \cite{meindl2021}.
Solving engineering problems the traditional way may not be the best solution due to its non-triviality to complex systems and their mathematical modeling.
% With the amount of available data and power processing, many tasks may now be done with the AI aid \cite{pham1998}.

% In the SHM field, detecting failures and monitoring the structure is vital to prevent damages. 
% Installing sensors in the structure and send signals to a central processing is a common way to provide predictive maintenance for structures \cite{kahandawa2012,nagayama2007}.
% Accidents with wagons can be avoided by detecting cracks in railways.
% AI can be used to determine the failures and possible damages to determine if the railway is able to keep working.

% For the control engineering area, a control system to UAVs are necessary to get them to do their tasks properly.
% A traditional \emph{white box} method is certainly a rock-solid but not a trivial way to do that task. 
% With a \emph{black box} approach, it is possible to determine the control forces by having only the initial pose and the desired trajectory, instead of modeling mathematically from the scratch \cite{loyola-gonzalez2019,wu2016}.
% This way, for determined situations, all the mathematical modeling complexity can be replaced to an AI system that can predict the control forces by giving easy-to-get information.

% The use of AI techniques as a different approach to the traditional engineering problems can decrease efforts to complex systems and facilitate problems resolution, providing low-costs and faster solutions. 

% \section{Objective}

% To develop two AI algorithms based on ANN to apply in smart systems.
% The main goals are: (i) to determine the control forces used to move an UAV based on its initial pose and the desired trajectory; and (ii) to detect railways cracks through piezoelectric signal for SHM.