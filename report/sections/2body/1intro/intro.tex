\chapter{Introduction}

The use of \gls*{ai} is very present nowadays \citep{lee2020,poola2017,rabunal2006}. 
This area of statistics neither is new nor started just now with the autonomous cars and voice assistants \citep{muthukrishnan2020}, but it is clear that in the last years it has been increasingly gaining more popularity. 
This happens mainly because of the advances that the World Wide Web has been had over the years \citep{leiner2009,cohen-almagor2011}, since dial-up internet connection, back in the eighties, until now, with broadband internet and smartphones equipped with 5G connection. 
Another factor is that in the past, the cost to get a large capacity of storage memory was significantly more expensive than it is now, what makes today cheaper and easy to get memory to store information \citep{goda2012}. 
With the amount of data available, the evolution of  internet and storage capacity, now it is not difficult to obtain, keep and analyze databases to make decisions \citep{duan2019}.

\gls*{ai} application is everywhere and today, more than ever, it is easy to realize that. 
Either to get multimedia recommendations on streaming platforms, like occurs at Netflix, YouTube, Spotify, and so many others platforms \citep{chan-olmsted2019}, or to make predictions on the financial market and sports betting \citep{milana2021,kollar2021,hubacek2019}, \gls*{ai} is there behind the scenes making all the magic happen. 
Evidently there is nothing really magical about them, it is pure mathematics combined with a programming language that produces the algorithm capable of doing those things \citep{goodfellow2016,aurelien2022,raschka2015,raschka2022}.
The launch of ChatGPT--3, and shortly thereafter ChatGPT--4, has shown the power of those technologies and how they can change the way people do things \citep{biswas2023,biswas2023a, lund2023,baidoo-anu2023}.

Getting into the smart systems application, the use of \gls*{ai} is widely used to \gls*{shm}, which is heavily used in the aerospace and civil fields, \citep{azimi2020,ye2019}. 
The level and the complexity of the \gls*{ai} to be applied to monitor the structure, whether is going to use \gls*{dl} and \gls*{nn} or simpler methods of \gls*{ml} like regressions, is determined by the problem itself and the results desired \citep{farrar2012}. 
In some cases, the standards methods use numerical techniques and they may not be feasible, especially when there is a huge data to be analyzed. 
Thus, taking the \gls*{ai} road is an alternative to get the needed results for the monitoring in a more practical way \citep{smarsly2007,sun2020}. 

Still in this context, but in the field of \gls*{uav}, the use of \gls*{ai} can be combined to integrate \gls*{uav} through wireless communication networks \citep{lahmeri2021} what can be useful in the agriculture sphere \citep{ahirwar2019} with technologies like \gls*{iot} \citep{verdouw2016,tzounis2017}.
Also, the use of the \gls*{ai} can be subtle, such as the use of a built-in \matlab function to make a simple \gls*{nn} to determine the final pose of a \gls*{uav} based on the initial pose and the forces applied on it \citep{geronel2023}, or can be more sophisticated, like the use of \gls*{ml} and \gls*{dl} algorithms to predict materials properties, design new materials, discover new mechanisms and control real dynamic systems \citep{guo2021,assilian1974}.

It is clear, therefore, that \gls*{ai} can transit into different fields, such as entertainment, business, health care, marketing, financial, agriculture, engineering, among others \citep{ruiz-real2020,yu2018,davenport2019,verma2021,mhlanga2020,pannu2015,ghatrehsamani2023}. 
The use of the \gls*{bd} can not only make it clear the scenario to be studied, but also to support making strategical decisions \citep{jeble2018,koscielniak2015}.
The internet and hardware improvement \citep{baji2018}, alongside the facility to storage data with accessible costs, encourages the \gls*{ai} use due to the benefits it can provide.

\section{Motivation}

With the 4.0 industry, the engineering evolution is growing bigger every year \citep{meindl2021}.
Solving engineering problems the traditional way may not be the best solution due to its non-triviality to complex systems and their mathematical modeling.
With the amount of available data and power processing, many tasks may now be done with the \gls*{ai} aid.

In the \gls*{shm} field, detecting failures and monitoring the structure is vital to prevent damages. 
Accidents can be prevented by detecting cracks in railways.
Installing sensors in the structure and send signals to a central processing is a common way to provide predictive maintenance for structures.
\gls*{ai} can be used to determine the failures and possible damages to determine IF the railway is able to keep working.

On the other hand, a control system to \glspl*{uav} are necessary to get them to do their tasks properly.
A traditional \emph{white box} method is certainly a rock-solid but not a trivial way to do that task. 
With a \emph{black box} approach, it is possible to determine the control forces by having only the initial pose and the desired trajectory, instead of modeling mathematically from the scratch.
This way, for determined situations, all the mathematical modeling complexity can be replaced to an \gls*{ai} system that can predict the control forces by giving easy-to-get information.

The use of \gls*{ai} techniques as a different approach to the traditional engineering problems can decrease efforts to complex systems and facilitate problems resolution, providing low-costs and faster solutions. 

\section{Objective}

To develop two \gls*{ai} algorithms based on \gls*{nn} to apply in smart systems.
The main goals are: (i) to determine the control forces used to move an \gls*{uav} based on its initial pose and the desired trajectory; and (ii) to detect railways cracks through piezoelectric signal for \gls*{shm}.