\section{Unnamed Aerial Vehicle}

\subsection{Usage}

An \gls*{uav} has several applications, going from the simplest to the mos sophisticated. It can be used since for entertainment, like toys; can be used commercially, for recording big shows in arenas, for example; and can be used in engineering, aiding in various context to improve some processing.

Due to its portability and autonomy, it can be used to facilitate the delivery o medicines. In this sense, \gls*{uav} can be used for transportation of medical goods in critical times, where other means of transportation may not be feasible.
In the final of 2019, COVID-19 pandemics spread throughout the world, making it difficult to deliver patients their needed medicines \cite{ramakrishnan2023,mcphillips2022}.
Besides, risks are inherent to the transportation and in come countries, like the USA, \gls*{uav} usage may be restricted \citep{thiels2015}.

In the agriculture context, in order to boost the productivity, \gls*{uav} can be used to remotely sense the farming, obtaining information on the state of the fields with non-contact procedures, like nutrient evaluation and soil monitoring; or even for aerial spraying, using pesticide to prevent damages in the plantation \citep{delcerro2021}.

The main reason for its adoptions is the mobility, low maintenance costs, hovering capacity, ease of deployment, etc. 
It is widely used for the civil infrastructure, gathering photographs faster than  satellite imagery and with better quality. 
Combining those benefits with \gls*{ai} can be a powerful tool for the future. \citet{sivakumar2021}. 

\subsection{Motion Equation}

Considering the \gls*{uav} a quadcopter, as the~\cref{fig:quadcopter_forces_scheme} shows, \citet{fossen1994} described the equation of motion as being
%
\begin{equation}
    \symbfup{M}_{\eta}(\symbfup{\eta})\ddot{\symbfup{\eta}} +
    \symbfup{C}_{\eta}(\symbfup{\nu},\symbfup{\eta})\dot{\symbfup{\eta}} +
    \symbfup{g}_{\eta}(\symbfup{\eta}) +
    \symbfup{K}_{\eta}(\symbfup{\eta})\symbfup{\eta} =
    \symbfup{\tau}_{\eta}(\symbfup{\eta}) + 
    \symbfup{F}_d
    \label{eq:uav_motion_equation}
\end{equation}
%
where 
\begin{description}[labelsep=1.7cm]
    \item[\(\symbfup{M}_{\eta}(\symbfup{\eta})\)] is the inertial matrix;
    \item[\(\symbfup{C}_{\eta}(\symbfup{v},\symbfup{\eta})\)] is the Coriolis matrix;
    \item[\(\symbfup{g}_{\eta}(\symbfup{\eta})\)] is the gravitational vector;
    \item[\(\symbfup{K}_{\eta}(\symbfup{\eta})\)] is the stiffness matrix;
    \item[\(\symbfup{\tau}_{\eta}(\symbfup{\eta})\)] is the control torque;
    \item[\(\symbfup{F}_d\)] is the gust vector; and
    \item[\(\symbfup{\nu}\)] is the velocity generalized coordinate in the body-frame.
\end{description}
%
\begin{figure}[!htb]
    \centering
    \includesvg{figures/3review/uav/uav_forces.svg}
    \caption[Quadcopter Dynamic Scheme]{Quadcopter Dynamic Scheme. \(F_i\,\) and \(T_i\), for \(i=1,2,3,4\), are the forces and the torque applied in the propeller, respectively. \(\omega_j\) and \(v_j\), for \(j=x,y,z\), are the momentum and the velocities applied in the \gls*{uav}, respectively.}
    \label{fig:quadcopter_forces_scheme}
\end{figure}

As detailed by \citet{geronel2023}, the matrices of the~\cref{eq:uav_motion_equation} are
%
\begin{align}
    \symbfup{M}_{\eta}(\symbfup{\eta}) &= \symbfup{I}_{6\times6} \\
    \symbfup{C}_{\eta}(\symbfup{\nu},\symbfup{\eta}) &= \symbfup{J}\symbfup{M}^{-1}\symbfup{C}\symbfup{J}^{-1} - \dot{\symbfup{J}}\symbfup{J}^{-1} \\
    \symbfup{g}_{\eta}(\symbfup{\eta}) &= \symbfup{J}\symbfup{M}^{-1}\symbfup{g}_{0} \\
    \symbfup{K}_{\eta}(\symbfup{\eta}) &= \symbfup{J}\symbfup{M}^{-1}\symbfup{K} \\
    \symbfup{\tau}_{\eta}(\symbfup{\eta}) &= \symbfup{J}\symbfup{M}^{-1}\symbfup{\tau}
    \label{eq:matrices_dynamic}
\end{align}
%
where \(\symbfup{J}\) is the derivative of the transformation matrix with respect to time. The main matrices of the~\cref{eq:matrices_dynamic} are
%
\begin{align}
    \symbfup{M} &= \diag(m,m,m,I_{xx},I_{yy},I_{zz})
    \label{eq:diag_M} \\
    \symbfup{C} &= \begin{bmatrix}
        0 & -m\omega_z & m\omega_y & 0 & 0 & 0 \\
        m\omega_z & 0 & -m\omega_x 0 & 0 & 0 & 0\\
        -m\omega_y & m\omega_x & 0 & 0 & 0 & 0 \\
        0 & 0 & 0 & 0 & 0 & (I_{zz}-I_{yy})\omega_y \\
        0 & 0 & 0 & (I_{xx}-I_{zz})\omega_z & 0 & 0 \\
        0 & 0 & 0 & 0 & (I_{yy}-I_{xx})\omega_x& 0 \\
    \end{bmatrix}
    \label{eq:coriolis_matrix} \\
    \symbfup{J} &= \begin{bmatrix}
        \symbfup{J}(\symbfup{\eta}) & \mathbf{0}_{6\times 1} \\
        \symbfup{j}_n & c\theta c\phi
    \end{bmatrix}
    \label{eq:transformation}
\end{align}
%
where