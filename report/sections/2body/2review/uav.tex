\section{Unnamed Aerial Vehicle}

\subsection{Usage}

An \gls*{uav} has several applications, going from the simplest to the most sophisticated. It can be used since for entertainment, like toys; commercially, to record big shows in arenas; surveillance, to monitor places; and also in engineering, aiding in various context to improve some processing.

Due to its portability and autonomy, it can be used to facilitate the delivery o medicines. In this sense, \gls*{uav} can be used for transportation of medical goods in critical times, where other means of transportation may not be feasible.
In the final of 2019, COVID-19 pandemics spread throughout the world, making it difficult to deliver patients their needed medicines \cite{ramakrishnan2023,mcphillips2022}.
Besides, risks are inherent to the transportation and in come countries, like the USA, \gls*{uav} usage may be restricted \citep{thiels2015}. 
A strategical way to use them is also welcome.

In the agriculture context, in order to boost the productivity, \gls*{uav} can be used to remotely sense the farming, obtaining information on the state of the fields with non-contact procedures, like nutrient evaluation and soil monitoring; or even for aerial spraying, using pesticide to prevent damages in the plantation \citep{delcerro2021}.

The main reason for its adoptions is the mobility, low maintenance costs, hovering capacity, ease of deployment, etc. 
It is widely used for the civil infrastructure, gathering photographs faster than  satellite imagery and with better quality. 
Combining those benefits with \gls*{ai} can be a powerful tool for the future. \citet{sivakumar2021}. 

\subsection{Motion Equation}

Considering the \gls*{uav} a quadcopter, as the~\cref{fig:quadcopter_forces_scheme} shows, it is assumed a defined coordinate system fixed in the body (\(Z_M\)). \citet{geronel2023}, based on the work of \citet{fossen1994}, described the equation of motion for a quadcopter with a payload as being:
%
\begin{equation}
    \symbfup{M}_{\eta_{c}}(\symbfup{\eta}_c)\ddot{\symbfup{\eta}}_c +
    \symbfup{C}_{\eta_c}(\symbfup{\nu},\symbfup{\eta}_c)\dot{\symbfup{\eta}_c} +
    \symbfup{g}_{\eta_c}(\symbfup{\eta}_c) +
    \symbfup{K}_{\eta_c}(\symbfup{\eta}_c)\symbfup{\eta}_c =
    \symbfup{\tau}_{\eta_c}(\symbfup{\eta}_c) + 
    \symbfup{F}_d
    \label{eq:uav_motion_equation}
\end{equation}
%
where \(\symbfup{M}_{\eta_{c}}(\symbfup{\eta}_c)\) is the inertial matrix; \(\symbfup{C}_{\eta_c}(\symbfup{\nu},\symbfup{\eta}_c)\) is the Coriolis matrix; \(\symbfup{g}_{\eta_c}(\symbfup{\eta}_c)\) is the gravitational vector; \(\symbfup{K}_{\eta_c}(\symbfup{\eta}_c)\) is the stiffness matrix; \(\ \symbfup{\tau}_{\eta_c}(\symbfup{\eta}_c)\) is the control torque; \(\symbfup{F}_d\) is the gust vector; and \(\symbfup{\nu}\) is the velocity generalized coordinate in the body-frame. All matrices and the development of the equations are explicit in the \citet{geronel2023} work.
%
\begin{figure}[!htb]
    \centering
    \caption[Quadcopter Dynamic Scheme]{Quadcopter Dynamic Scheme. \(F_i\,\) and \(T_i\), (\(i=1,2,3,4\)), are the forces and the torque applied in the propeller, respectively. \(\omega_j\) and \(v_j\), (\(j=x,y,z\)), are the momentum and the velocities applied in the \gls*{uav}, respectively.}
    \includesvg{figures/3review/uav/uav_forces.svg}
    
    {\footnotesize Source: prepared by the author.}
    \label{fig:quadcopter_forces_scheme}
\end{figure}
%
% The generalized coordinate's vector of the quadrotor in the inertial reference frame is
% %
% \begin{equation}
%     \symbfup{\eta} = \begin{Bmatrix}
%         x & y & z & \theta & \phi & \psi
%     \end{Bmatrix}^{\intercal}
% \end{equation}
% %
% where \(x,\ y,\ z\) and \(\phi,\ \theta,\ \psi\) are the positions and orientation angles of the quadrotor.
% As detailed by \citet{geronel2023}, the matrices of the~\cref{eq:uav_motion_equation} are
% %
% \begin{align}
%     \symbfup{M}_{\eta}(\symbfup{\eta}) &= \symbfup{I}_{6\times6} \\
%     \symbfup{C}_{\eta}(\symbfup{\nu},\symbfup{\eta}) &= \symbfup{J}\symbfup{M}^{-1}\symbfup{C}\symbfup{J}^{-1} - \dot{\symbfup{J}}\symbfup{J}^{-1} \\
%     \symbfup{g}_{\eta}(\symbfup{\eta}) &= \symbfup{J}\symbfup{M}^{-1}\symbfup{g}_{0} \\
%     \symbfup{\tau}_{\eta}(\symbfup{\eta}) &= \symbfup{J}\symbfup{M}^{-1}\symbfup{\tau}
%     \label{eq:matrices_dynamic}
% \end{align}
% %
% where \(\symbfup{J}\) is the derivative of the transformation matrix with respect to time. All of them are explicit in the appendix of the work of \citet{geronel2023}. Note that in his work, the matrices refers to a quadcopter with a payload attached to it, while in the present work the payload is disconsidered, hence all the terms related to the payload is null.
% %
% \begin{align}
%     \symbfup{M} &= \diag(m,m,m,I_{xx},I_{yy},I_{zz})
%     \label{eq:diag_M} \\
%     \symbfup{C} &= \begin{bmatrix}
%         0 & -m\omega_z & m\omega_y & 0 & 0 & 0 \\
%         m\omega_z & 0 & -m\omega_x 0 & 0 & 0 & 0\\
%         -m\omega_y & m\omega_x & 0 & 0 & 0 & 0 \\
%         0 & 0 & 0 & 0 & 0 & I_1\omega_y \\
%         0 & 0 & 0 & I_2\omega_z & 0 & 0 \\
%         0 & 0 & 0 & 0 & I_3\omega_x& 0 \\
%     \end{bmatrix}
%     \label{eq:coriolis_matrix}
% \end{align}
% %
% where \(I_1\) \(=I_{zz}-I_{yy}\) is the moment of inertia along the \(x\) axis; \(I_2\) \(=I_{xx}-I_{zz}\) is the moment of inertia along the \(y\) axis;\(I_3\) \(=I_{yy}-I_{xx}\) is the moment of inertia along the \(z\) axis; \(\symbfup{g}_0\) \(= \begin{bmatrix} -mgs\theta & mg\theta s\phi & mg\theta \phi & 0 & 0 & 0\end{bmatrix}^{\intercal}\) is the gravitational vector.


% A new transformation matrix \(\symbfup{J}\) is considered by using the relationship \(\symbfup{\eta} = \symbfup{J}\symbfup{\nu}\), where \(\symbfup{J}\) is defined as
% %
% \begin{equation}
%     \symbfup{J} = \begin{bmatrix}
%         \symbfup{J}(\symbfup{\eta}) & \mathbf{0}_{6\times 1} \\
%         \symbfup{j}_n & \theta\phi
%     \end{bmatrix}
%     \label{eq:transformation}
% \end{equation}
% %
% and \(\symbfup{J}(\symbfup{\eta})\) is defined as
% %
% \begin{equation}
%     \symbfup{J}(\symbfup{\eta}) = \begin{bmatrix}
%         \psi \theta & s\phi s\theta \psi - \phi s\psi & \phi s\theta \psi + s \psi s\phi & 0 & 0 & 0 \\
%         s\psi \theta & s\phi s\theta s\psi + \phi \psi & \phi s\theta s \psi - s\phi \psi & 0 & 0 & 0 \\
%         -s\theta & s\phi \theta & \phi \theta & 0 & 0 & 0 \\
%         0 & 0 & 0 & 1 & s\phi t\theta & \phi t\theta \\
%         0 & 0 & 0 & 0 & \phi & -s\phi \\
%         0 & 0 & 0 & 0 & \frac{s\phi}{\theta} & \frac{\phi}{\theta}
%     \end{bmatrix}
% \end{equation}
% %
% where \(\symbfup{j}_n = \begin{bmatrix}-s\theta & \theta s\phi & 0 & 0 & 0 & 0 \end{bmatrix}\).

\subsection{Control Algorithm}

\citet{geronel2023} also developed a \matlab algorithm to control the quadrotor. It can control it in three different trajectories: rectangular, circular and linear.
Given the \(\symbf{\tau}\) as the input vector, which represents the Position Controller \(U_1(t)\) and the Attitude Controller \(U_2(t),\ U_3(t),\ U_4(t)\), it is able to give a complete overview of the quadrotor's motion.
%
\begin{equation}
    \symbf{\tau} = \begin{Bmatrix}
        U_1 & U_2 & U_3 & U_4
    \end{Bmatrix}^{\intercal}
\end{equation}

The algorithm provides the state-space vector \(\symbfup{x}_s\) with the quadrotor position and angles, as their derivatives.
%
\begin{equation}\setcounter{MaxMatrixCols}{13}
    \symbf{x}_s =
    \begin{bmatrix}
        x&y&z&\phi&\theta&\psi&\dot{x}&\dot{y}&\dot{z}&\dot{\phi}&\dot{\theta}&\dot{\psi}
    \end{bmatrix}        
    \label{eq:xs_vector_matlab}
\end{equation}