\section{Softwares}

\subsection{PyTorch}

A framework is a group of libraries for a programming language and there is a lot of deep learning ones available, probably the most popular ones are TensorFlow and PyTorch. 
While the first one was developed by Google and released in 2015 the second one was developed by Meta (Facebook), although it is now under the Linux Foundation umbrella, and released in 2016, being both open-source.
Many companies, like Uber \citep{goodman2017} and Tesla \citep{pytorch2019}, use PyTorch in their \gls*{ai} team.


Aiming to be ease of use, PyTorch provides a pythonic programming style, that is, with previous knowledge of Python, programming with PyTorch is natural, since it follows the standard code style of other Python codes.
Besides, its focus is in the researcher, since it lets the complexity inherent to machine learning handled by the library itself, but it provides tools that allow researchers to manually control the execution of their code, allowing to make their own customization to gain more performance \citep{paszke2019}.

The consequential ease of use must be followed by performance to be useful, hence PyTorch implementation accepts additional complexity to provide performance, when it is needed.
The simplicity of the internal implementation of PyTorch allows to implements additional features, depending on the researcher wish \citep{paszke2019}.

Therefor, choosing PyTorch is a strategical approach in machine learning, because it not only allows to naturally code in Python standards, but also due to its flexibility and modularity.

\subsection{Matlab}

The standard in the engineering industry, \matlab is a powerful toolbox that...