\section{Introdução}

Diferentes abordagens são utilizadas para controle de sistemas eletro-mecânicos, como o modelo paramétrico caixa preta, em que determina-se as forças de controle através da posição inicial e a trajetória desenvolvida.
Desta forma, é possível determinar as forças de controle de um VANT utilizando técnicas de aprendizado de máquina.

\section{Objetivo}

Desenvolver uma rede neural que determine as forças de controle a partir da trajetória do VANT.

\section{Materiais e Métodos}

A determinação das forças de controle a partir da trajetória é realizada a partir de uma rede neural.
Esta é uma das técnicas de aprendizado de máquinas que permite reconhecimento de padrões em dados. A Fig.~\ref{fig:rede_neural} mostra esquematicamente o funcionamento de uma rede neural.\cite{haykin1999}
%
\begin{figure}[H]
    \centering
    \caption{Modelo de rede neural.}
    \includesvg[width=\columnwidth]{../cic_2023/figures/nn.svg}

    {\footnotesize Fonte: próprio autor.}
    \label{fig:rede_neural}
\end{figure}
%
Um modelo paramétrico caixa branca foi desenvolvido para determinar a trajetória de um VANT.\cite{geronel2023}.
O objetivo da rede é fazer o papel de uma função inversa ao modelo caixa branca. 
As matrizes das forças de controle e de espaço de estado são, respectivamente:
%
\begin{align}
    \tau &= \begin{bmatrix}
        U_1 & U_2 & U_3 & U_4 
    \end{bmatrix} \\ 
    \setcounter{MaxMatrixCols}{20}
    \mathbf{x}_s &= \begin{bmatrix}
        x & y & z & \theta & \phi & \psi & \dot{x} & \dot{y} & \dot{z} & \dot{\theta} & \dot{\phi} & \dot{\psi}
    \end{bmatrix}
\end{align}
%
Uma rede neural do tipo \emph{multi-layer perceptron} foi designada para realizar o treinamento nos dados.

\section{Resultados e Discussão}

A Fig.~\ref{fig:comparison} mostra uma comparação entre um valor real das forças de controle normalizada e as forças de controle normalizadas provenientes da rede neural relativa à mesma trajetória.
%
\begin{figure}[H]
    \centering
    \caption{Comparação entre a previsão do modelo e o valor real}
    \resizebox{.7\columnwidth}{!}{
    \import{/home/gabriel/Documentos/deep-learning/presentations/cic_2023/figures/}{comparison_default_normal.pgf}}

    {\footnotesize Fonte: próprio autor.}

    \label{fig:comparison}
\end{figure}
%
É possível perceber que para \(U_i\) \((i = 1, 2, 3, 4)\), a rede neural conseguiu reconhecer os padrões e determinar as forças de controle.
Vale notar que, apesar de \(U_1\) parecer estar discrepante na diferença entre os valores reais e os previstos, a escala do gráfico fortalece quem essa percepção.

\section{Conclusão}

A rede neural conseguiu determinar as forças de controle normalizadas de forma satisfatŕoia. Os próximos passos são: desenvolver um algoritmo para desnormalizar a matriz de saída da rede; sofisticar a rede neural; e simular as trajetórias com os valores normalizados e desnormalizados obtidos pela rede.
%
\bibliographystyle{acm}
\bibliography{ref}

