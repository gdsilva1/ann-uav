\section{Contexto do estudo}

\begin{frame}{Sistema Inteligente}
\[\boxed{\text{\alert<2>{Engenharia}} + \text{Inteligência artificial} = \text{Sistema Inteligente}}\]
\end{frame}
% ===========================
\begin{frame}{Monitoramento da Integridade Estrutural (SHM)}
\begin{block}{Finalidade}
    Diagnóstico e análise de uma estrutura.
\end{block} \pause
\vfill
\begin{itemize}
    \item Mecânica
    \item Civil
    \item Aeroespacial/Aeronáutica
\end{itemize} \pause
\vfill
Sensores \(\rightarrow\) Sistema central \(\rightarrow\) Análise dos dados \(\rightarrow\) Decisão
\end{frame}
% ===========================
\begin{frame}{Monitoramento da Integridade Estrutural (SHM)}
Métodos utilizados:
\begin{itemize}
    \item Acelerômetros
    \item Inspeção gráfica \(\rightarrow\) câmeras digitais
    \item \alert<2>{Sensores piezoelétricos}
\end{itemize}
\end{frame}
% ===========================
\begin{frame}{Controle do VANT}
\pause
\begin{block}{O que temos?}
    \begin{itemize}
        \item Algoritmo de controle do VANT.
        \item Implementação em MATLAB.
        \item Trajetórias: retangular, circular e linear.
    \end{itemize}
\end{block} \pause
\vfill
\begin{figure}
    \centering
    \includesvg{figures/dinamica_vant.svg}
    \end{figure}
\end{frame}

\begin{frame}{Controle do VANT}

\begin{block}{O que será feito?}
    \begin{itemize}
        \item Algoritmo para determinar as forças usadas.
        \item Dados de entrada: trajetória e posição inicial.
        \item Redes neurais.
    \end{itemize} 
\end{block}
\vfill
\begin{figure}
\centering
\includesvg{figures/dinamica_vant.svg}
\end{figure}
\end{frame}