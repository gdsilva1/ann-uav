\section{Equações}

\begin{frame}{Equações}

A definição de derivada é dada por:

\begin{equation}
    \dv{y}{x} = \lim_{\Delta x\to 0} \frac{f(x + \Delta x) - f(x)}{\Delta x}
\end{equation}

Além disso, a transformada de Laplace é:

\begin{equation}
    \mathcal{L}\{f(t)\} = \int_0^{\infty} f(t) \, e^{-st} \, \dd{t}
\end{equation}

Uma função do segundo grau tem a forma \(f(x) = ax^2+bx+c\), com \(a,b,c \in \mathbb{R}\) constantes.

Em trigonometria, a expressão s \(\sin^2x+\cos^2x = 1\) é muito útil.
    
\end{frame}

\section{Blocos}

\begin{frame}{Blocos}
    \begin{block}{Bloco padrão}
        \begin{itemize}
            \item Este é um bloco
            \item Este é um bloco
        \end{itemize}
    \end{block}

    \begin{alertblock}{Bloco de alerta}
        \begin{enumerate}
            \item Este é um bloco de alerta
            \item Este é um bloco de alerta    
        \end{enumerate}
    \end{alertblock}

    \begin{exampleblock}{Bloco de exemplo}
        a$a$b$b$c$c$d$d$e$e$f$f$g$g$h$h$i$i$j$j$k$k$l$l$m$m$n$n$o$o$p$p$q$q$r$r$s$s$t$t$u$u$v$v$w$w$x$x$y$y$z$z$
    \end{exampleblock}

\end{frame}

\begin{frame}
    The limit of a function when some number tends to another is denoted inline by l \(\lim_{x\to 0} f(x)\).
\end{frame}