% =========================================================
% =========================================================
\begin{frame}{Introdução}
\begin{itemize}
    \item[\bullet] Na área de sistemas inteligentes, o monitoramento dos mesmos se faz, diversas vezes, através de \emph{inteligência artificial}.

    \item[\bullet] A utilização de técnicas de Machine Learning pode ser utilizada para situações como monitoramento da integridade estrutural (SHM) e controle/automação de sistemas, por exemplo.

    \item[\bullet] Uma das aplicações, é para auxiliar no próprio controle de vôo de um VANT, controlando incertezas a partir de determinadas trajetórias.
\end{itemize}
\end{frame}
% =========================================================
% =========================================================
\begin{frame}{Metodologia}
\begin{columns}[t]
    \column{0.4\textwidth}
A partir das forças de controle, o script fornece:
%
\begin{equation}
    F(\symbfup{\tau}_{4\times n}) = \symbfup{X}_{12\times n}
\end{equation}

A rede neural deverá fazer o papel de uma função inversa:
\begin{equation}
    F^{-1}(\symbfup{X}_{12\times n}) = \symbfup{\tau}_{4\times n}
\end{equation}
\scriptsize{\(n\): cada instante de tempo a partir da discretização.}
\column{0.6\textwidth}
Inicialmente, problema foi tratado como \alert{regressão} e uma rede neural do tipo multi-layer percepetron foi desenvolvida. 
%
\begin{figure}
    \centering
    \includesvg[scale=0.5, pretex=\tiny]{../../report/figures/3review/nn/nn.svg}
\end{figure}
\end{columns}
\end{frame}

% =========================================================
% =========================================================
\begin{frame}{Resultados}
\begin{figure}
\centering
\resizebox{\textwidth}{!}{%
\import{/home/gabriel/Documentos/deep-learning/presentations/cic2023/figures}{comparison_default_beamer.pgf}}
\end{figure}
\end{frame}

% =========================================================
% =========================================================
\begin{frame}{Considerações finais}
\begin{itemize}
    \item[\bullet] A rede neural recebe e retorna apenas matrizes normalizadas.
    \item[\bullet] A matriz de controle se aproximou de forma satisfatória para todos os termos.
\end{itemize}
%
\begin{block}{Trabalhos futuros}
\begin{itemize}
    \item[\bullet] Desnormalizar a matriz de saída da rede neural.
    \item[\bullet] Sofisticar a rede neural.
    \item[\bullet] Simular as trajetórias a partir das forças de controle geradas pela rede neural.
\end{itemize}
\end{block}
\end{frame}