% ==============================================================================
\begin{frame}{Introdução}
    
\begin{itemize}
    \item Modelo paramétrico caixa preta.
    \item Determinação das forças de controle a partir da posição inicial e trajetória.
    \item Utilização de redes neurais.
\end{itemize}

\begin{block}{Objetivo}
    Desenvolver uma rede neural para determinar as forças de controle de um VANT a partir de sua trajetória para auxiliar no controle do mesmo.
\end{block}
\end{frame}

% ==============================================================================
\begin{frame}{Materiais e Métodos}
\begin{columns}
\column{0.55\textwidth}
\begin{itemize}
    \item A determinação das forças de controle será determinada a partir de uma \alert{rede neural}.
    \item Algoritmo do modelo paramétrico de caixa branca.
    \item As forças de controle são:
    \begin{equation*}
        \tau = \begin{bmatrix}
            U_1 & U_2 & U_3 & U_4
        \end{bmatrix}^{\intercal}
    \end{equation*}
    \item O vetor de estado é:
    \begin{equation*}
        \setcounter{MaxMatrixCols}{20}
        \mathbf{x}_s = \begin{bmatrix}
            x & y & z & \theta & \phi & \psi & \dot{x} & \dot{y} & \dot{z} & \dot{\theta} & \dot{\phi} & \dot{\psi}
        \end{bmatrix}^{\intercal}
    \end{equation*}
    \item Uma rede neural do tipo \emph{multi-layer perceptron} foi designada para realizar o treinamento nos dados.
\end{itemize}
\column{0.45\textwidth}
\begin{figure}[H]
    \centering
    \includesvg[width=\columnwidth, pretex=\footnotesize]{../../../report/figures/3review/nn/nn2.svg}
\end{figure}
\end{columns}
\end{frame}

% ==============================================================================
\begin{frame}{Resultados e Discussão}
\begin{figure}[H]
    \centering
    \resizebox{\columnwidth}{!}{
    \import{/home/gabriel/Documentos/deep-learning/report/figures/4results/uav/}{comparison_dark_background_beamer.pgf}}

    \label{fig:comparison}
\end{figure}
\end{frame}

% ==============================================================================
\begin{frame}{Conclusão}
A rede neural consegue determinar as forças de controle normalizadas de forma satisfatória.

\begin{block}{Trabalho Futuro}
\begin{itemize}
    \item Desenvolver um algoritmo para desnormalizar a matriz de saída da rede.
    \item Simular as trajetórias com os valores normalizados e desnormalizados obtidos pela rede.
    \item Sofisticar a rede neural conforme a necessidade.
\end{itemize}
\end{block}

\end{frame}